%!TEX root = ../main_text.tex
\let\cleardoublepage\clearpage
\titlepage[Universidade Federal de Ouro Preto]{

\vspace{-21cm}
\textbf{UNIVERSIDADE FEDERAL DE OURO PRETO}

\vspace{13cm}

Orientador: Dr. Ricardo Augusto Rabelo Oliveira \\ \vskip8pt

\vspace{3cm}

\hspace{7.5cm}
\parbox{0.5\textwidth}{
Dissertação submetida ao Instituto
de Ciências Exatas e Biológicas da Universidade
Federal de Ouro Preto para qualificação ao
título de Mestre em Ciência da Computação
}

\vspace{2.8cm}

\begin{center}
Ouro Preto, Setembro de 2017
\end{center}
}

\let\cleardoublepage\clearpage

\titlepage[Universidade Federal de Ouro Preto]{
\vspace{-21cm}
\textbf{}
\vspace{13cm}

Orientador: Dr. Ricardo Augusto Rabelo Oliveira \\ \vskip8pt

\vspace{3cm}

\begin{figure}[h!]
  \begin{center}
    \includegraphics[width=1.0\textwidth]{img/logo.png}
    \label{fig:goal}
  \end{center}
\end{figure}

}

%\begin{figure}[h]
 % \begin{center}
  %  \includegraphics[width=1.0\textwidth]{img/ataDefesa.pdf}
   % \label{fig:ataDefesa}
  %\end{center}
%\end{figure}

\begin{comment}

\dedication{\vspace{4cm}
\begin{flushright}
Dedico este trabalho a você que sempre me fez acreditar na realização dos meus sonhos e trabalhou muito para que eu pudesse realizá-los, \emph{Mãe}. A minha \emph{família} e \emph{amigos} que sempre me apoiaram nas horas difíceis. Ao meu \emph{pai}, que mesmo distante influenciou para que pudesse chegar até aqui.

\end{flushright}

}
\end{comment}

%% Abstract in portuguese
\begin{abstract}[\smaller \titulo\\ \vspace*{1cm} \smaller Resumo]
  %\thispagestyle{empty}


Este trabalho constitui-se de uma abordagem do problema de particionamento \hs\ para dispositivos vestíveis com foco em aumento de performance, visando o gasto de recursos em seu \textit{trade-off}.
A tecnologia digital de fatores como microeletrônica, sensores e comunicação móvel são constantemente melhoradas à medida que a informação torna-se cada vez mais sem-fio, tornando-a um grande estímulo para o desenvolvimento de dispositivos inteligentes e conectados como sistemas embarcados IoT ou vestíveis (do inglês \wearables). 
Isso é visto no rápido desenvolvimento de dispositivos para comércio, entretanto, ainda com a dificuldade de satisfazer os requerimentos das várias aplicações modernas. 
Tais dispositivos utilizam de um leque enorme de sensores e necessitam de um serviço autônomo, o que implica numa grande demanda de performance somado com o baixo consumo de energia sem deixar design, segurança e confiabilidade a desejar. 
Para obter um produto de alta qualidade, atendendo aos requisitos solicitados, esta pesquisa consiste na análise de projetos para sistemas \wearables\ utilizando plataformas FPGA como meio na realização de particionamentos \hs\ obtendo um bom \speedup\ e eficiência em comparação a outros sistemas como \textit{in-software}. 
%Com mais pesquisas voltadas para saúde e conforto humano, os dispositivos \wearables\ estão possuindo cada vez mais habilidades para detectar e identificar pessoas ou seus comportamentos e assim tornar um complemento/auxílio para as atividades cotidianas.

%To adequately address these demands, sophisticated embedded computing and embedded design technologies are needed. After an introduction to modern mobile systems, this paper discusses the huge heterogeneous area of these systems, and considers serious issues and challenges in their design. Subsequently, it discusses the embedded computing and design technologies needed to adequately address the issues and overcome the challenges in order to satisfy the stringent requirements of the modern mobile systems.
Palavras-chave: Particionamento \hs, Sistema \Wearable, Sistemas Embarcados.
\end{abstract}
\let\cleardoublepage\clearpage


%% Abstract
\begin{abstract}[\smaller An Approach of Hardware and Software Partitioning Problem for the Design of Wearable Computing Systems\\ \vspace*{1cm} \smaller Abstract]

  %\thispagestyle{empty}
  
This work consists of an approach of the hardware and software partitioning problem for wearable devices focused on performance enhancement, aiming at the expense of resources in their trade-off.
The digital technology of factors such as microelectronics, sensors and mobile communication are constantly improved as information becomes increasingly wireless, making it a great stimulus for the development of intelligent and connected devices like embedded IoT or wearable systems.
This is seen in the rapid development of devices for commerce, however, still with the difficulty of satisfying the requirements of the various modern applications.
Such devices use a wide range of sensors and require autonomous service, which implies a great demand for performance coupled with low power consumption without leaving design, security and reliability to be desired.
In order to obtain a high-quality product, meeting the requested requirements, this research consists of the analysis of designs for wearables systems using FPGA platforms as a means of realizing hardware and software partitions obtaining good speedup and efficiency compared to other systems such as in-software.

Keywords: Hardware and Software Partitioning, Wearable System, Embedded System.
\end{abstract}

\begin{comment}

\let\cleardoublepage\clearpage
%% Declaration
\begin{declaration}
Esta dissertação é resultado de meu próprio trabalho, exceto onde referência explícita é feita ao trabalho de outros, e não foi submetida para outra qualificação nesta nem em outra universidade.
\vspace*{1cm}
\begin{flushright}
Danilo Santos Souza
\end{flushright}
\end{declaration}

%% Acknowledgements
\let\cleardoublepage\clearpage
\begin{acknowledgements}

Primeiramente agradeço à \emph{Deus} por me amparar nos momentos difíceis e me dar força para superar as dificuldades.

Aos meus pais, \emph{Gilsa e Denildo}, e a meu irmão \emph{Diego}, pelo incentivo, apoio e afeto.

A minha avó \emph{Carmelita}, pelo exemplo de dedicação e por ser uma mãe pra mim.

Ao meu Tio \emph{Denilson}, por sempre acreditar em mim quando alguns não acreditavam, e por ser um exemplo de pai e homem.

À toda minha \emph{família}, pelo carinho e força.

Agradeço ao meu orientador \emph{Gustavo Peixoto Silva} pela oportunidade concedida, pela impecável orientação e pelo apoio no desenvolvimento do trabalho. Agradeço ainda ao meu Coorientador \emph{Haroldo Gambini Santos} pelas valiosas contribuições para o presente trabalho.

Aos meus amigos \emph{Bruna, Priscilla, Gabriela, Thayane, Lorran, Aristides, Alisson, Pereira, Lucas, Marco, Emilia, Danielle e Flavio} pela ajuda, pelas longas horas de estudo, amizade, carinho e força, fundamentais para que eu conseguisse chegar até aqui.

À República \emph{Navio Pirata}, Moradores e Ex-alunos, por me acolherem, pelo companheirismo e amizade, tornando esta minha segunda casa.

Ao \emph{CNPq}, à \emph{FAPEMIG}, à \emph{CAPES} e à \emph{UFOP} pelo apoio recebido no desenvolvimento deste trabalho.

Ao \emph{PPGCC - UFOP}, professores e técnicos, em especial aos professores Fabrício, Gustavo e Haroldo, e a Mariana pela ajuda sempre que fez-se necessário.

Enfim, agradeço a \emph{todos} que, de alguma forma, acreditaram e torceram por mim, participaram de minha vida e ajudaram na realização deste trabalho.

\end{acknowledgements}

\end{comment}


%% Preface
%\begin{preface}
%
%\end{preface}

% ToC
\tableofcontents
\listoffigures
\let\cleardoublepage\clearpage
%\listoftables
%%\listofalgorithms
%\listofalgorithmes

\begin{comment}

%% Strictly optional!
\frontquote%
{Talvez não tenha conseguido fazer o melhor, mas lutei para que o melhor fosse feito. Não sou o que deveria ser, mas Graças a Deus, não sou o que era antes.}%
{Marthin Luther King}

\end{comment}
